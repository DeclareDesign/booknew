% Options for packages loaded elsewhere
\PassOptionsToPackage{unicode}{hyperref}
\PassOptionsToPackage{hyphens}{url}
%
\documentclass[
]{article}
\usepackage{lmodern}
\usepackage{amssymb,amsmath}
\usepackage{ifxetex,ifluatex}
\ifnum 0\ifxetex 1\fi\ifluatex 1\fi=0 % if pdftex
  \usepackage[T1]{fontenc}
  \usepackage[utf8]{inputenc}
  \usepackage{textcomp} % provide euro and other symbols
\else % if luatex or xetex
  \usepackage{unicode-math}
  \defaultfontfeatures{Scale=MatchLowercase}
  \defaultfontfeatures[\rmfamily]{Ligatures=TeX,Scale=1}
\fi
% Use upquote if available, for straight quotes in verbatim environments
\IfFileExists{upquote.sty}{\usepackage{upquote}}{}
\IfFileExists{microtype.sty}{% use microtype if available
  \usepackage[]{microtype}
  \UseMicrotypeSet[protrusion]{basicmath} % disable protrusion for tt fonts
}{}
\makeatletter
\@ifundefined{KOMAClassName}{% if non-KOMA class
  \IfFileExists{parskip.sty}{%
    \usepackage{parskip}
  }{% else
    \setlength{\parindent}{0pt}
    \setlength{\parskip}{6pt plus 2pt minus 1pt}}
}{% if KOMA class
  \KOMAoptions{parskip=half}}
\makeatother
\usepackage{xcolor}
\IfFileExists{xurl.sty}{\usepackage{xurl}}{} % add URL line breaks if available
\IfFileExists{bookmark.sty}{\usepackage{bookmark}}{\usepackage{hyperref}}
\hypersetup{
  pdftitle={Planning},
  hidelinks,
  pdfcreator={LaTeX via pandoc}}
\urlstyle{same} % disable monospaced font for URLs
\usepackage[margin=1in]{geometry}
\usepackage{color}
\usepackage{fancyvrb}
\newcommand{\VerbBar}{|}
\newcommand{\VERB}{\Verb[commandchars=\\\{\}]}
\DefineVerbatimEnvironment{Highlighting}{Verbatim}{commandchars=\\\{\}}
% Add ',fontsize=\small' for more characters per line
\usepackage{framed}
\definecolor{shadecolor}{RGB}{248,248,248}
\newenvironment{Shaded}{\begin{snugshade}}{\end{snugshade}}
\newcommand{\AlertTok}[1]{\textcolor[rgb]{0.94,0.16,0.16}{#1}}
\newcommand{\AnnotationTok}[1]{\textcolor[rgb]{0.56,0.35,0.01}{\textbf{\textit{#1}}}}
\newcommand{\AttributeTok}[1]{\textcolor[rgb]{0.77,0.63,0.00}{#1}}
\newcommand{\BaseNTok}[1]{\textcolor[rgb]{0.00,0.00,0.81}{#1}}
\newcommand{\BuiltInTok}[1]{#1}
\newcommand{\CharTok}[1]{\textcolor[rgb]{0.31,0.60,0.02}{#1}}
\newcommand{\CommentTok}[1]{\textcolor[rgb]{0.56,0.35,0.01}{\textit{#1}}}
\newcommand{\CommentVarTok}[1]{\textcolor[rgb]{0.56,0.35,0.01}{\textbf{\textit{#1}}}}
\newcommand{\ConstantTok}[1]{\textcolor[rgb]{0.00,0.00,0.00}{#1}}
\newcommand{\ControlFlowTok}[1]{\textcolor[rgb]{0.13,0.29,0.53}{\textbf{#1}}}
\newcommand{\DataTypeTok}[1]{\textcolor[rgb]{0.13,0.29,0.53}{#1}}
\newcommand{\DecValTok}[1]{\textcolor[rgb]{0.00,0.00,0.81}{#1}}
\newcommand{\DocumentationTok}[1]{\textcolor[rgb]{0.56,0.35,0.01}{\textbf{\textit{#1}}}}
\newcommand{\ErrorTok}[1]{\textcolor[rgb]{0.64,0.00,0.00}{\textbf{#1}}}
\newcommand{\ExtensionTok}[1]{#1}
\newcommand{\FloatTok}[1]{\textcolor[rgb]{0.00,0.00,0.81}{#1}}
\newcommand{\FunctionTok}[1]{\textcolor[rgb]{0.00,0.00,0.00}{#1}}
\newcommand{\ImportTok}[1]{#1}
\newcommand{\InformationTok}[1]{\textcolor[rgb]{0.56,0.35,0.01}{\textbf{\textit{#1}}}}
\newcommand{\KeywordTok}[1]{\textcolor[rgb]{0.13,0.29,0.53}{\textbf{#1}}}
\newcommand{\NormalTok}[1]{#1}
\newcommand{\OperatorTok}[1]{\textcolor[rgb]{0.81,0.36,0.00}{\textbf{#1}}}
\newcommand{\OtherTok}[1]{\textcolor[rgb]{0.56,0.35,0.01}{#1}}
\newcommand{\PreprocessorTok}[1]{\textcolor[rgb]{0.56,0.35,0.01}{\textit{#1}}}
\newcommand{\RegionMarkerTok}[1]{#1}
\newcommand{\SpecialCharTok}[1]{\textcolor[rgb]{0.00,0.00,0.00}{#1}}
\newcommand{\SpecialStringTok}[1]{\textcolor[rgb]{0.31,0.60,0.02}{#1}}
\newcommand{\StringTok}[1]{\textcolor[rgb]{0.31,0.60,0.02}{#1}}
\newcommand{\VariableTok}[1]{\textcolor[rgb]{0.00,0.00,0.00}{#1}}
\newcommand{\VerbatimStringTok}[1]{\textcolor[rgb]{0.31,0.60,0.02}{#1}}
\newcommand{\WarningTok}[1]{\textcolor[rgb]{0.56,0.35,0.01}{\textbf{\textit{#1}}}}
\usepackage{graphicx,grffile}
\makeatletter
\def\maxwidth{\ifdim\Gin@nat@width>\linewidth\linewidth\else\Gin@nat@width\fi}
\def\maxheight{\ifdim\Gin@nat@height>\textheight\textheight\else\Gin@nat@height\fi}
\makeatother
% Scale images if necessary, so that they will not overflow the page
% margins by default, and it is still possible to overwrite the defaults
% using explicit options in \includegraphics[width, height, ...]{}
\setkeys{Gin}{width=\maxwidth,height=\maxheight,keepaspectratio}
% Set default figure placement to htbp
\makeatletter
\def\fps@figure{htbp}
\makeatother
\setlength{\emergencystretch}{3em} % prevent overfull lines
\providecommand{\tightlist}{%
  \setlength{\itemsep}{0pt}\setlength{\parskip}{0pt}}
\setcounter{secnumdepth}{-\maxdimen} % remove section numbering
\usepackage{booktabs}
\usepackage{longtable}
\usepackage{array}
\usepackage{multirow}
\usepackage{wrapfig}
\usepackage{float}
\usepackage{colortbl}
\usepackage{pdflscape}
\usepackage{tabu}
\usepackage{threeparttable}
\usepackage{threeparttablex}
\usepackage[normalem]{ulem}
\usepackage{makecell}
\usepackage{xcolor}

\title{Planning}
\author{}
\date{\vspace{-2.5em}}

\begin{document}
\maketitle

\hypertarget{planning}{%
\section{Planning}\label{planning}}

When embarking on a prospective research design, it is becoming standard
practice in many research communities to publicly register a
pre-analysis plans (PAP) prior to the implementation of some or all of
the Data strategy. PAPs serve many functions, but most importantly, they
clarify which design choices were made before data collection and which
were made afterward. Sometimes -- perhaps everytime! -- we conduct a
research study, aspects of \(M\), \(I\), \(D\), and \(A\) shift along
the way. A concern is that they shift in ways that invalidate the
apparent conclusions of the study. For example, ``\(p\)-hacking'' is the
shady practice of trying out many regression specifications until the
\(p\)-value associated with an important test attains statistical
significance. PAPs help researchers to credibly communicate to skeptics
when design decisions were made: if the regression specification was
detailed in a PAP posted before any data were collected, the test cannot
be the result of a \(p\)-hack.

What belongs in a PAP? Thus far, the set of decisions that should be
specified in a PAP remains remarkably unclear. PAP templates and
checklists are proliferating, and the number of items they suggest range
from nine to sixty. PAPs themselves are becoming longer and more
detailed, with some in the American Economic Association (AEA) and
Evidence in Governance and Politics (EGAP) study registries reaching
hundreds of pages, as researchers seek to be ever more comprehensive.
Some registries emphasize the registration of the hypotheses to be
tested, while others emphasize the registration of the tests that will
used. We read many PAPs -- it is often hard to assess whether these
detailed plans actually contain the key analytically-relevant details.

Our view is that, minimally, a PAP should include a design declaration.
A good deal of the discussion of what goes in a PAP centers on the
answer strategy \(A\) -- what estimator to use, what covariates to
condition on, what subsets of the data to include. But of course we also
need to know the details of \(D\) -- how units were sampled, how
treatments were assigned, how the outcomes will be measured. We need to
know about \(I\) because we need to know what the target of inference
is\footnote{A major concern in medical trials is ``outcome switching,''
  wherein the eventual report focuses on different health outcomes that
  originally intended. When we switch outcomes, we switch inquiries!}.
We need enough of \(M\) to describe \(I\) in sufficient detail. In
short, a design declaration is what belongs in a PAP, because a design
declaration specifies all of the analytically-relevent design decisions.

In addition to a design declaration, a PAP should include mock analyses
conducted on simulated data. If the design declaration is done formally
in code, creating simulated data that resemble the eventual realized
data is quite straightforward. We think researchers should run their
Answer strategy on the mock data, creating mock figures and tables that
will eventually be made with real data. In our experience, \emph{this}
is the step that really causes researchers to think hard about all
aspects of their design.

Strictly speaking, preanalysis plans should include design declaration,
but they do not \emph{require} design diagnosis. But since the design
that is finally settled on as the design to be implemented is usually
chosen as the result of a diagnosis, it can be informative to describe,
in a preanalysis plan, the reasons why the particular design was chosen.
For this reason, a PAP might include estimates of diagnosands like
power, rmse, or bias. If a researcher writes in a PAP that the power to
detect a very small effect is large, then if the study comes back null,
the eventual writeup can much more credibly rule out ``low power'' as an
explanation for the null. Moreover, ex ante design diagnosis
communicates the assumptions under which they thought the design was a
good one before they ran the study. These reasons are often the basis on
which we convince skeptics of the value of the design, so writing them
down \emph{before} the results are known increases the faith we put in
them.

\hypertarget{example}{%
\subsection{Example}\label{example}}

\begin{itemize}
\tightlist
\item
  This section will be an example PAP.
\item
  We will take a (relatively simple) PAP from the EGAP website (with
  permission) and declare the design in code.
\item
  We will make fake figures and tables using mock data
\item
  We will choose wisely, because we will use the same study for the
  section on reconciliation. We want there to be a few differences
  (enough to make the point) but that the differences don't shame the
  author
\item
  Very open to suggestions on a good study.
\end{itemize}

Possibility: Bonilla and Tillery (2020) from the APSR on the effects of
intersectional framing on support for the BLM movement. The takeaway is
that the feminist and LGBTQ frames cause a decrease in support among the
men in the sample, but not the women.

This is a 4-arm (control vs.~three treatments) experiment among Black
americans recruited via qualtrics. 4 dependent variables. Almost
everything they do in the article and appendix was pre-registered. They
don't do everything they said they would do (het fx by linked fate) but
close. This study seems complex enough to be worth doing, but simple
enough to make it seem ``easy'' to do design declaration. A takeaway is
that the study was probably underpowered for most of the het fx
estimation, a point that gets conceeded post-hoc in the article.

The data are posted on dataverse, so the reconciliation section will
work ok too.

\begin{Shaded}
\begin{Highlighting}[]
\NormalTok{model <-}\StringTok{ }
\StringTok{  }\KeywordTok{declare_population}\NormalTok{(}
    \DataTypeTok{N =} \DecValTok{800}\NormalTok{,}
    \DataTypeTok{female =} \KeywordTok{rbinom}\NormalTok{(N, }\DecValTok{1}\NormalTok{, }\DataTypeTok{prob =} \FloatTok{0.51}\NormalTok{),}
    \DataTypeTok{lgbtq =} \KeywordTok{rbinom}\NormalTok{(N, }\DecValTok{1}\NormalTok{, }\DataTypeTok{prob =} \FloatTok{0.05}\NormalTok{),}
    \DataTypeTok{age =} \KeywordTok{sample}\NormalTok{(}\DecValTok{18}\OperatorTok{:}\DecValTok{80}\NormalTok{, N, }\DataTypeTok{replace =} \OtherTok{TRUE}\NormalTok{),}
    \DataTypeTok{religiosity =} \KeywordTok{sample}\NormalTok{(}\DecValTok{1}\OperatorTok{:}\DecValTok{6}\NormalTok{, N, }\DataTypeTok{replace =} \OtherTok{TRUE}\NormalTok{),}
    \DataTypeTok{income =} \KeywordTok{sample}\NormalTok{(}\DecValTok{1}\OperatorTok{:}\DecValTok{12}\NormalTok{, N, }\DataTypeTok{replace =} \OtherTok{TRUE}\NormalTok{),}
    \DataTypeTok{college =} \KeywordTok{rbinom}\NormalTok{(N, }\DecValTok{1}\NormalTok{, }\DataTypeTok{prob =} \FloatTok{0.5}\NormalTok{),}
    \DataTypeTok{blm_familiarity =} \KeywordTok{sample}\NormalTok{(}\DecValTok{1}\OperatorTok{:}\DecValTok{4}\NormalTok{, N, }\DataTypeTok{replace =} \OtherTok{TRUE}\NormalTok{),}
    \DataTypeTok{linked_fate =} \KeywordTok{sample}\NormalTok{(}\DecValTok{1}\OperatorTok{:}\DecValTok{5}\NormalTok{, N, }\DataTypeTok{replace =} \OtherTok{TRUE}\NormalTok{, }\DataTypeTok{prob =} \KeywordTok{c}\NormalTok{(}\FloatTok{0.05}\NormalTok{, }\FloatTok{0.05}\NormalTok{, }\FloatTok{0.15}\NormalTok{, }\FloatTok{0.25}\NormalTok{, }\FloatTok{0.5}\NormalTok{)),}
    \DataTypeTok{U =} \KeywordTok{runif}\NormalTok{(N),}
    \DataTypeTok{blm_support_latent =} \KeywordTok{rescale}\NormalTok{(}
\NormalTok{      U }\OperatorTok{+}\StringTok{ }\FloatTok{0.1} \OperatorTok{*}\StringTok{ }\NormalTok{blm_familiarity }\OperatorTok{+}\StringTok{ }\FloatTok{0.45} \OperatorTok{*}\StringTok{ }\NormalTok{linked_fate }\OperatorTok{+}\StringTok{ }\FloatTok{0.001} \OperatorTok{*}\StringTok{ }\NormalTok{age }\OperatorTok{+}\StringTok{ }
\StringTok{        }\FloatTok{0.25} \OperatorTok{*}\StringTok{ }\NormalTok{lgbtq }\OperatorTok{+}\StringTok{ }\FloatTok{0.01} \OperatorTok{*}\StringTok{ }\NormalTok{income }\OperatorTok{+}\StringTok{ }\FloatTok{0.1} \OperatorTok{*}\StringTok{ }\NormalTok{college }\OperatorTok{+}\StringTok{ }\FloatTok{-0.1} \OperatorTok{*}\StringTok{ }\NormalTok{religiosity)}
\NormalTok{  ) }\OperatorTok{+}\StringTok{ }
\StringTok{  }\KeywordTok{declare_potential_outcomes}\NormalTok{(}
    \DataTypeTok{blm_support_Z_general =} \KeywordTok{likert_cut}\NormalTok{(blm_support_latent),}
    \DataTypeTok{blm_support_Z_nationalism =} \KeywordTok{likert_cut}\NormalTok{(blm_support_latent }\OperatorTok{+}\StringTok{ }\FloatTok{0.01} \OperatorTok{+}\StringTok{ }\NormalTok{linked_fate }\OperatorTok{*}\StringTok{ }\FloatTok{0.01}\NormalTok{),}
    \DataTypeTok{blm_support_Z_feminism =} \KeywordTok{likert_cut}\NormalTok{(blm_support_latent }\OperatorTok{+}\StringTok{ }\FloatTok{0.01} \OperatorTok{+}\StringTok{ }\NormalTok{female }\OperatorTok{*}\StringTok{ }\FloatTok{0.05}\NormalTok{),}
    \DataTypeTok{blm_support_Z_intersectional =} \KeywordTok{likert_cut}\NormalTok{(blm_support_latent }\OperatorTok{+}\StringTok{ }\FloatTok{0.01} \OperatorTok{+}\StringTok{ }\NormalTok{lgbtq }\OperatorTok{*}\StringTok{ }\FloatTok{0.15}\NormalTok{)}
\NormalTok{  )}

\NormalTok{inquiry <-}\StringTok{  }
\StringTok{  }\KeywordTok{declare_estimands}\NormalTok{(}
    \DataTypeTok{ATE_nationalism =} \KeywordTok{mean}\NormalTok{(blm_support_Z_nationalism }\OperatorTok{-}\StringTok{ }\NormalTok{blm_support_Z_general),}
    \DataTypeTok{ATE_feminism =} \KeywordTok{mean}\NormalTok{(blm_support_Z_feminism }\OperatorTok{-}\StringTok{ }\NormalTok{blm_support_Z_general),}
    \DataTypeTok{ATE_intersectional =} \KeywordTok{mean}\NormalTok{(blm_support_Z_intersectional }\OperatorTok{-}\StringTok{ }\NormalTok{blm_support_Z_general),}
    \DataTypeTok{DID_nationalism_linked_fate =} 
      \KeywordTok{cov}\NormalTok{(blm_support_Z_nationalism }\OperatorTok{-}\StringTok{ }\NormalTok{blm_support_Z_general, linked_fate)}\OperatorTok{/}\KeywordTok{var}\NormalTok{(linked_fate),}
    \DataTypeTok{DID_feminism_gender =} 
      \KeywordTok{cov}\NormalTok{(blm_support_Z_feminism }\OperatorTok{-}\StringTok{ }\NormalTok{blm_support_Z_general, female)}\OperatorTok{/}\KeywordTok{var}\NormalTok{(female),}
    \DataTypeTok{DID_intersectional_lgbtq =} 
      \KeywordTok{cov}\NormalTok{(blm_support_Z_intersectional }\OperatorTok{-}\StringTok{ }\NormalTok{blm_support_Z_general, lgbtq)}\OperatorTok{/}\KeywordTok{var}\NormalTok{(lgbtq)}
\NormalTok{  )}
\end{Highlighting}
\end{Shaded}

\begin{Shaded}
\begin{Highlighting}[]
\NormalTok{data_strategy <-}\StringTok{ }
\StringTok{  }\KeywordTok{declare_assignment}\NormalTok{(}\DataTypeTok{conditions =} \KeywordTok{c}\NormalTok{(}\StringTok{"general"}\NormalTok{, }\StringTok{"nationalism"}\NormalTok{, }\StringTok{"feminism"}\NormalTok{, }\StringTok{"intersectional"}\NormalTok{), }\DataTypeTok{simple =} \OtherTok{TRUE}\NormalTok{) }\OperatorTok{+}\StringTok{ }
\StringTok{  }\KeywordTok{declare_reveal}\NormalTok{(blm_support, Z) }
\end{Highlighting}
\end{Shaded}

\begin{Shaded}
\begin{Highlighting}[]
\NormalTok{answer_strategy <-}
\StringTok{  }\KeywordTok{declare_estimator}\NormalTok{(}
\NormalTok{    blm_support }\OperatorTok{~}\StringTok{ }\NormalTok{Z,}
    \DataTypeTok{term =} \KeywordTok{c}\NormalTok{(}\StringTok{"Znationalism"}\NormalTok{, }\StringTok{"Zfeminism"}\NormalTok{, }\StringTok{"Zintersectional"}\NormalTok{),}
    \DataTypeTok{model =}\NormalTok{ lm_robust,}
    \DataTypeTok{estimand =} \KeywordTok{c}\NormalTok{(}\StringTok{"ATE_nationalism"}\NormalTok{, }\StringTok{"ATE_feminism"}\NormalTok{, }\StringTok{"ATE_intersectional"}\NormalTok{),}
    \DataTypeTok{label =} \StringTok{"unadjusted"}
\NormalTok{  ) }\OperatorTok{+}
\StringTok{  }\KeywordTok{declare_estimator}\NormalTok{(}
\NormalTok{    blm_support }\OperatorTok{~}\StringTok{ }\NormalTok{Z }\OperatorTok{+}\StringTok{ }\NormalTok{age }\OperatorTok{+}\StringTok{ }\NormalTok{female }\OperatorTok{+}\StringTok{ }\KeywordTok{as.factor}\NormalTok{(linked_fate) }\OperatorTok{+}\StringTok{ }\NormalTok{lgbtq,}
    \DataTypeTok{term =} \KeywordTok{c}\NormalTok{(}\StringTok{"Znationalism"}\NormalTok{, }\StringTok{"Zfeminism"}\NormalTok{, }\StringTok{"Zintersectional"}\NormalTok{),}
    \DataTypeTok{estimand =} \KeywordTok{c}\NormalTok{(}\StringTok{"ATE_nationalism"}\NormalTok{, }\StringTok{"ATE_feminism"}\NormalTok{, }\StringTok{"ATE_intersectional"}\NormalTok{),}
    \DataTypeTok{model =}\NormalTok{ lm_robust,}
    \DataTypeTok{label =} \StringTok{"adjusted"}
\NormalTok{  ) }\OperatorTok{+}
\StringTok{  }\KeywordTok{declare_estimator}\NormalTok{(}
\NormalTok{    blm_support }\OperatorTok{~}\StringTok{ }\NormalTok{Z }\OperatorTok{*}\StringTok{ }\NormalTok{linked_fate,}
    \DataTypeTok{term =} \StringTok{"Zfeminism:linked_fate"}\NormalTok{,}
    \DataTypeTok{model =}\NormalTok{ lm_robust,}
    \DataTypeTok{estimand =} \StringTok{"DID_nationalism_linked_fate"}\NormalTok{,}
    \DataTypeTok{label =} \StringTok{"het-fx-nationalism-linked-fate"}
\NormalTok{  ) }\OperatorTok{+}
\StringTok{  }\KeywordTok{declare_estimator}\NormalTok{(}
\NormalTok{    blm_support }\OperatorTok{~}\StringTok{ }\NormalTok{Z }\OperatorTok{*}\StringTok{ }\NormalTok{female,}
    \DataTypeTok{term =} \StringTok{"Zfeminism:female"}\NormalTok{,}
    \DataTypeTok{model =}\NormalTok{ lm_robust,}
    \DataTypeTok{estimand =} \StringTok{"DID_feminism_gender"}\NormalTok{,}
    \DataTypeTok{label =} \StringTok{"het-fx-feminism-female"}
\NormalTok{  ) }\OperatorTok{+}
\StringTok{  }\KeywordTok{declare_estimator}\NormalTok{(}
\NormalTok{    blm_support }\OperatorTok{~}\StringTok{ }\NormalTok{Z }\OperatorTok{*}\StringTok{ }\NormalTok{lgbtq,}
    \DataTypeTok{term =} \StringTok{"Zfeminism:lgbtq"}\NormalTok{,}
    \DataTypeTok{model =}\NormalTok{ lm_robust,}
    \DataTypeTok{estimand =} \StringTok{"DID_intersectional_lgbtq"}\NormalTok{,}
    \DataTypeTok{label =} \StringTok{"het-fx-intersectional-lgbtq"}
\NormalTok{  )}
\end{Highlighting}
\end{Shaded}

\begin{Shaded}
\begin{Highlighting}[]
\NormalTok{design <-}\StringTok{ }\NormalTok{model }\OperatorTok{+}\StringTok{ }\NormalTok{inquiry }\OperatorTok{+}\StringTok{ }\NormalTok{data_strategy }\OperatorTok{+}\StringTok{ }\NormalTok{answer_strategy}
\end{Highlighting}
\end{Shaded}

\begin{tabular}{l|l|r|r|r}
\hline
Estimand & Estimator & Bias & RMSE & Power\\
\hline
ATE feminism & ATE feminism & 0.00 & 0.06 & 0.76\\
\hline
ATE feminism & ATE feminism & 0.00 & 0.08 & 0.47\\
\hline
ATE intersectional & ATE intersectional & 0.00 & 0.06 & 0.24\\
\hline
ATE intersectional & ATE intersectional & 0.00 & 0.08 & 0.14\\
\hline
ATE nationalism & ATE nationalism & 0.00 & 0.05 & 0.97\\
\hline
ATE nationalism & ATE nationalism & 0.00 & 0.08 & 0.74\\
\hline
DID feminism gender & DID feminism gender & 0.00 & 0.16 & 0.27\\
\hline
DID intersectional lgbtq & DID intersectional lgbtq & -0.52 & 0.67 & 0.07\\
\hline
DID nationalism linked fate & DID nationalism linked fate & -0.05 & 0.07 & 0.06\\
\hline
\end{tabular}

\begin{table}
\begin{center}
\begin{tabular}{l c c c c}
\hline
 & Model 1 & Model 2 & Model 3 & Model 4 \\
\hline
(Intercept)                      & $3.63^{***}$ & $1.35^{***}$  & $1.43^{***}$ & $3.06^{***}$ \\
                                 & $(0.06)$     & $(0.11)$      & $(0.15)$     & $(0.12)$     \\
Znationalism                     & $0.36^{***}$ & $0.25^{***}$  & $0.09$       & $0.70^{***}$ \\
                                 & $(0.08)$     & $(0.05)$      & $(0.23)$     & $(0.18)$     \\
Zfeminism                        & $0.19^{*}$   & $0.16^{***}$  & $0.03$       & $0.26$       \\
                                 & $(0.08)$     & $(0.05)$      & $(0.22)$     & $(0.19)$     \\
Zintersectional                  & $0.12$       & $0.06$        & $0.15$       & $0.54^{**}$  \\
                                 & $(0.08)$     & $(0.05)$      & $(0.19)$     & $(0.19)$     \\
female                           &              & $0.06$        &              &              \\
                                 &              & $(0.03)$      &              &              \\
lgbtq                            &              & $0.43^{***}$  &              &              \\
                                 &              & $(0.10)$      &              &              \\
age                              &              & $0.00$        &              &              \\
                                 &              & $(0.00)$      &              &              \\
religiosity                      &              & $-0.14^{***}$ &              &              \\
                                 &              & $(0.01)$      &              &              \\
income                           &              & $0.02^{***}$  &              &              \\
                                 &              & $(0.00)$      &              &              \\
college                          &              & $0.13^{***}$  &              &              \\
                                 &              & $(0.03)$      &              &              \\
linked\_fate                     &              & $0.54^{***}$  & $0.54^{***}$ &              \\
                                 &              & $(0.02)$      & $(0.04)$     &              \\
blm\_familiarity                 &              & $0.12^{***}$  &              & $0.23^{***}$ \\
                                 &              & $(0.02)$      &              & $(0.04)$     \\
Znationalism:linked\_fate        &              &               & $0.05$       &              \\
                                 &              &               & $(0.05)$     &              \\
Zfeminism:linked\_fate           &              &               & $0.04$       &              \\
                                 &              &               & $(0.05)$     &              \\
Zintersectional:linked\_fate     &              &               & $-0.01$      &              \\
                                 &              &               & $(0.05)$     &              \\
Znationalism:blm\_familiarity    &              &               &              & $-0.13^{*}$  \\
                                 &              &               &              & $(0.07)$     \\
Zfeminism:blm\_familiarity       &              &               &              & $-0.04$      \\
                                 &              &               &              & $(0.07)$     \\
Zintersectional:blm\_familiarity &              &               &              & $-0.17^{*}$  \\
                                 &              &               &              & $(0.07)$     \\
\hline
R$^2$                            & $0.02$       & $0.69$        & $0.56$       & $0.07$       \\
Adj. R$^2$                       & $0.02$       & $0.69$        & $0.56$       & $0.06$       \\
Num. obs.                        & $800$        & $800$         & $800$        & $800$        \\
RMSE                             & $0.86$       & $0.48$        & $0.58$       & $0.84$       \\
\hline
\multicolumn{5}{l}{\scriptsize{$^{***}p<0.001$; $^{**}p<0.01$; $^{*}p<0.05$}}
\end{tabular}
\caption{Statistical models}
\label{table:coefficients}
\end{center}
\end{table}

\includegraphics{02_Planning_files/figure-latex/unnamed-chunk-13-1.pdf}

\hypertarget{debates-over-the-value-of-paps}{%
\subsection{Debates over the value of
PAPs}\label{debates-over-the-value-of-paps}}

PAPs are often described as a tool for tying researchers' hand and
reducing ``researcher degrees of freedom'' to seek out congenial
analyses. PAPs are sometimes criticized because (A) they don't
\emph{actually} constrain what researchers do and (B) they
\emph{shouldn't}.

PAPs might not actually constrain researchers because most journals and
pressess do not check against PAPs. As reviewers, we have sometimes
requested that authors send in the PAPs referenced in their papers. In
nearly every case, some analyses promised in the PAP were not included,
even in an appendix. Some analyses that appear in the paper were not
included in the PAP. As we discuss in greater detail in section XXX on
reconciliation, current practice is clearly too causal when
distinguishing which analyses were pre-specified and which were not.

Some critics of PAPs charge that we shouldn't pre-register our analysis
plans because we should be open to discovery. We learn once we arrive in
the field what the interesting questions are, so we shouldn't be
constrained by what we thought would happen when sitting in our offices.
Furthermore, sometimes people pre-register analyses that are biased,
misspecified, or are otherwise inappropriate, so they shouldn't be
required to present them. We agree with all the above points --
researchers should create post-implementation reports (CITE JENS in PA)
that follow the PAP. Analyses over and above the PAP should simply be
labeled as such.

Our take is that PAPs are helpful tools for researchers to plan research
better and they are useful for clarifying what the researcher was
thinking at each stage.

\hypertarget{other-benefits-of-paps}{%
\subsection{Other benefits of PAPS}\label{other-benefits-of-paps}}

\begin{itemize}
\tightlist
\item
  involving partners: agreeing on the analysis procedure ex ante reduces
  ex post conflict
\item
  frontloading research design decisions (get more specific, identify
  problems)
\item
  Anticipating what might go wrong: attrition, noncompliance, study
  failure, missing covariate data, etc.
\item
  Faithful reanalysis. Reanalysts lose a degree of freedom in
  determining what an author might have intended by a given analysis.
\item
  Easier design comparison. If a design is declared at the preanalysis
  plan phase, then it enables direct comparison with the design as
  implemented in the final write-up. Side-by-side comparison of the code
  neatly clarifies which design choices were made ex-ante and ex-post.
  Side-by-side comparison of the performance of a planned and
  implemented designs clarify under what conditions deviations from
  plans are defensible improvements.
\item
  ethics (summarize and cite Jay's ideas here:
  \url{http://www.jasonlyall.com/wp-content/uploads/2020/08/PreregisterYourEthics.pdf})
\item
  ethical outcomes are potential outcomes, so we need to think about
  them ex ante not just on the basis of revealed outcomes
\end{itemize}

\hypertarget{countering-objections}{%
\subsubsection{countering objections:}\label{countering-objections}}

\begin{itemize}
\item
  Time-consuming. Yes but in our experience we only pay this cost for
  failed studies. Fur successful studies, nearly all work that goes in
  to the pap pays off in terms of higher qualtity design, literal words
  we already wrote, and written-in-advance analysis code.
\item
  Won't stop determined cheaters. Yes -- remember that the goal is not
  to prevent fraud, it's to help \emph{researchers} improve their
  designs. Science depends on trust.
\item
  Replication is better (Coffman and Niederle (2015)). These are
  complements, not substitutes.
\item
  I can't preregister what I will do because I don't know what I will
  find. That's fine too, just write down how you will go about
  ``finding'' things so we (and YOU) can understand your own process.
\end{itemize}

\hypertarget{other-thoughts}{%
\subsubsection{Other thoughts}\label{other-thoughts}}

\begin{itemize}
\tightlist
\item
  when is the right moment to write a pap?
\item
  relationship to registered reports?
\item
  SOPs
\end{itemize}

\hypertarget{citations-on-paps}{%
\subsubsection{citations on paps}\label{citations-on-paps}}

\begin{itemize}
\tightlist
\item
  Casey, Glennerster, and Miguel (2012): early entry
\item
  Olken (2015): halfway skeptical
\item
  Green and Lin (2016): Standard operating procedures
\item
  Christensen and Miguel (2018): review
\item
  Coffman and Niederle (2015): a skeptical take (prefer replications).
\item
  Humphreys, Sierra, and Windt (2013): nonbinding
\item
  Miguel et al. (2014): distinguishes between disclosure and PAP
\item
  Ofosu and Posner (2020): apparently paps hinder publication?
\end{itemize}

\hypertarget{refs}{}
\leavevmode\hypertarget{ref-bonilla_tillery_2020}{}%
Bonilla, Tabitha, and Alvin B. Tillery. 2020. ``Which Identity Frames
Boost Support for and Mobilization in the \#Blacklivesmatter Movement?
An Experimental Test.'' \emph{American Political Science Review} 114
(4): 947--62. \url{https://doi.org/10.1017/S0003055420000544}.

\leavevmode\hypertarget{ref-Casey2012}{}%
Casey, Katherine, Rachel Glennerster, and Edward Miguel. 2012.
``Reshaping Institutions: Evidence on Aid Impacts Using a Pre-Analysis
Plan.'' \emph{The Quarterly Journal of Economics} 127 (4): 1755--1812.

\leavevmode\hypertarget{ref-Christensen2018}{}%
Christensen, Garret, and Edward Miguel. 2018. ``Transparency,
Reproducibility, and the Credibility of Economics Research.''
\emph{Journal of Economic Literature} 56 (3): 920--80.
\url{https://doi.org/10.1257/jel.20171350}.

\leavevmode\hypertarget{ref-Coffman2015}{}%
Coffman, Lucas C., and Muriel Niederle. 2015. ``Pre-Analysis Plans Have
Limited Upside, Especially Where Replications Are Feasible.'' \emph{The
Journal of Economic Perspectives} 29 (3): 81--97.

\leavevmode\hypertarget{ref-Green2015}{}%
Green, Donald P., and Winston Lin. 2016. ``Standard Operating
Procedures: A Safety Net for Pre-Analysis Plans.'' \emph{PS: Political
Science \& Politics} 49 (3): 495--99.

\leavevmode\hypertarget{ref-Humphreys2013a}{}%
Humphreys, Macartan, Raul de la Sierra, and Peter van der Windt. 2013.
``Fishing, Commitment, and Communication: A Proposal for Comprehensive
Nonbinding Research Registration.'' \emph{Political Analysis} 21 (1):
1--20.

\leavevmode\hypertarget{ref-Miguel2014}{}%
Miguel, Edward, Colin Camerer, Katherine Casey, Joshua Cohen, Kevin M
Esterling, Alan Gerber, Rachel Glennerster, et al. 2014. ``Promoting
Transparency in Social Science Research.'' \emph{Science} 343 (6166):
30.

\leavevmode\hypertarget{ref-Ofosu2020}{}%
Ofosu, George K., and Daniel N. Posner. 2020. ``Do Pre-Analysis Plans
Hamper Publication?'' \emph{AEA Papers and Proceedings} 110 (May):
70--74.

\leavevmode\hypertarget{ref-Olken2015}{}%
Olken, Benjamin A. 2015. ``Promises and Perils of Pre-Analysis Plans.''
\emph{Journal of Economic Perspectives} 29 (3): 61--80.

\end{document}
